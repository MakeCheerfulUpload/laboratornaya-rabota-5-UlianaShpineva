\thispagestyle{fancy}
\fancyhf{}
\fancyfoot[R]{\thepage}
\renewcommand{\headrulewidth}{0pt}
\renewcommand{\footrulewidth}{0pt}
\footskip = 0pt

\title{lab6}
\author{uliana.shpineva }
\date{November 2022}


\setcounter{page}{49}

\begin{minipage}[bs]{0.45\textwidth}
\qquad Подготовка специалистов в области математики, механики и физики осуществляется на механико-математическом и физическом факультетах.

\qquad Студенты-математики специализируются по прикладной математике, функциональному анализу и теории функций, по дифференциальным и интегральным уравнениям.

\qquad Студенты-механики специализируются по аэро- и гидромеханике, а также по механике деформируемого твердого тела.

\qquad Студенты-физики специализируются по теоретической физике, оптике и спектроскопии, радиофизике и электронике, физике полупроводников и диэлектриков, физике твердого тела.

\qquad Ниже приведены варианты вступительного письменного экзамена по математике и задачи из билетов устного экзамена по физике на механико-математическом и физическом факультетах Куйбышевского университета в 1976 году.
\newline\\
\textbf{М а т е м а т и к а}\\

\textbf{Механико-математический факультет}

\qquad В а р и а н т\quad1

\qquad 1.Один из двух соосных конусов опирается вершиной на основание другого конуса, длина его образующей равна \textit{l}, вели-
\end{minipage}
\qquad
\begin{minipage}[bs]{0.45\textwidth}
\begin{minipage}[bs]{\textwidth}
\textbf{Физический факультет}

\qquad В а р и а н т\quad3

\qquad 1. Отношение длин двух отрезков, заключенных между параллельными плоскостями, равно \textit{k}, а величины углов, которые каждый из этих отрезков составляет с одной из плоскостей, относятся как 2:3. Найти величины этих углов и допустимые значения \textit{k}.
\end{minipage}

\qquad \raggedright{2. Решить уравнение}
$$\sqrt{x - 1} - \sqrt{x + 3 - 4\sqrt{x - 1}} = 2.$$
\qquad \raggedright{3. Решить уравнение}
$$\cos^{2}x - \cos^{2}3x + \cos^{2}2x = 0.$$
\qquad  \raggedright{4. Решить неравенство}
$$x^{-[\lg^{2}x + \lg x^{3} + 3] \log_{x}\sqrt{2}} \le \frac{x}{\sqrt{2}}.$$
\begin{minipage}[bs]{\textwidth}
\qquad В а р и а н т\quad4

\qquad1. В правильной четырехугольной пирамиде через два боковых ребра, не принадлежащих одной грани, проведена плоскость. Отношение площади сечения к площади боковой поверхности пирамиды равно \textit{k}. Найти величину угла между двумя смежными боковыми гранями и допустимые значения \textit{k}.

\qquad \raggedright{2. Решить уравнение\\}
$$2x + 1 - \sqrt{x^{2} - 3x + 1} = 0.$$
\end{minipage}
\end{minipage}

