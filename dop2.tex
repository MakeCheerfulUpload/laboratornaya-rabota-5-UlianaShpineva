\documentclass[12pt]{article}
\usepackage[utf8]{inputenc}
\usepackage[a4paper, top=15mm, right=10mm, bottom=15mm, left=5mm, headsep=-5mm]{geometry}
\usepackage[english,russian]{babel}
\usepackage{tikz}
\usepackage{fancyhdr}
\usepackage{lettrine} 
\usepackage{GoudyIn}
\usepackage{mathtools}
\usepackage{amssymb}
%\usepackage{graphicx, type1cm, lettrine}

\usetikzlibrary{calc}


\title{Lab6}
\author{Uliana Shpineva}
\date{November 2022}

\fancyheadoffset[R]{5cm}
\fancyhead[C]{96\qquad КНИГА III ПРЕДЛ. IV. ТЕОРЕМА\\}
\renewcommand{\footrulewidth}{0 mm}
\renewcommand{\headrulewidth}{0 mm}
\renewcommand\LettrineFontHook{\GoudyInfamily}
\fancyfoot{}


\begin{document}
\thispagestyle{fancy}
\begin{minipage}[H]{0.3\textwidth}
\begin{tikzpicture}[line width=2pt]
\coordinate (A) at (-29.5mm,-6mm);
\coordinate (B) at (-24mm,-18mm);
\coordinate (C) at ((22mm, -20mm);
\coordinate (D) at (29mm, -8mm);
\coordinate (E) at (intersection of A--C and B--D);
\coordinate (F) at (0,0);
\fill[blue!70] (E) -- ($(E)!0.5cm!(F)$) arc (85:11:0.5) -- cycle;
\fill[yellow] (E) -- ($(E)!0.5cm!(D)$) arc (12:-14:0.5) -- cycle;
\draw[draw=red] (B) node [below left]  {\scriptsize B} -- (D) node [right]  {\scriptsize D};
\draw (A) node [left]  {\scriptsize A} -- (C) node [below right]  {\scriptsize C};
\draw[draw=blue!70] (0, 0) circle (30mm);
\fill[black] (E) node [below] {\scriptsize E};
\draw[dashed] (E) -- (F) node [above] {\scriptsize F};
\end{tikzpicture}
\end{minipage}
\qquad\qquad
\begin{minipage}[bs]{0.55\textwidth}
\vspace{2cm}
\lettrine[lines=4]{E}{} \textit{сли в круге две прямые, не проходящие через центр, пересекаются, они не делят друг друга пополам.}\\

\qquad Если одна из прямых проходит через центр, очевидно, она ее не может рассекать пополам другая прямая,
не проходящая через центр.

\qquad Но если ни одна из прямых \tikz\draw[line width=2pt] (0,0.2cm) node [above] {\scriptsize A} -- (1.5cm,0.2cm) node [above]  {\scriptsize C}; или \tikz\draw[line width=2pt, draw=red] (0,0.2cm) node [above] {\scriptsize B} -- (1.5cm,0.2cm) node [above]  {\scriptsize D}; не проходит через центр, проведем \tikz\draw[line width=2pt, dashed] (0,0.2cm) node [above] {\scriptsize E} -- (1.5cm,0.2cm) node [above]  {\scriptsize F}; из центра
к точке их пересечения.

\begin{center}
Если \tikz\draw[line width=2pt] (0,0.2cm) node [above] {\scriptsize A} -- (1.5cm,0.2cm) node [above]  {\scriptsize C}; делится пополам,

\tikz\draw[line width=2pt, dashed] (0,0.2cm) node [above] {\scriptsize E} -- (1.5cm,0.2cm) node [above]  {\scriptsize F}; 
$\perp$ ей (пр. III.$_3$)

$\therefore$ \begin{tikzpicture}[line width=2pt]
\fill[blue!70] (0,-0.5) node [left] {\scriptsize E} -- (85:0)  node [above] {\scriptsize F} arc (85:11:0.5)  -- (0,-0.5); 
\fill[fill=yellow] (0,-0.5) -- (-41:0.6) arc (11:-14:0.5) -- cycle;
\fill[black] (-41:0.6) node [right] {\scriptsize C};
\end{tikzpicture}
$=$ 
\begin{tikzpicture}\draw[line width=1pt] (0,0) -- (0,0.5);
\draw[line width=2pt] (90:0.5) arc (90:180:0.5);
\draw[line width=1pt] (0,0) -- (-0.5,0);
\end{tikzpicture}

и если \tikz\draw[line width=2pt, draw=red] (0,0.2cm) node [above] {\scriptsize B} -- (1.5cm,0.2cm) node [above]  {\scriptsize D}; делится пополам,

\tikz\draw[line width=2pt, dashed] (0,0.2cm) node [above] {\scriptsize E} -- (1.5cm,0.2cm) node [above]  {\scriptsize F}; $\perp$ \tikz\draw[line width=2pt, draw=red] (0,0.2cm) node [above] {\scriptsize B} -- (1.5cm,0.2cm) node [above]  {\scriptsize D}; (пр. III.$_3$)

$\therefore$ \tikz\fill[blue!70] (0,-0.5) node [left] {\scriptsize E} -- (85:0)  node [above] {\scriptsize F} arc (85:11:0.5) node [below, right] {\scriptsize D}  -- (0,-0.5); $=$ 
\begin{tikzpicture}\draw[line width=1pt] (0,0) -- (0,0.5);
\draw[line width=2pt] (90:0.5) arc (90:180:0.5);
\draw[line width=1pt] (0,0) -- (-0.5,0);
\end{tikzpicture};

и $\therefore$ \tikz\fill[blue!70] (0,-0.5) node [left] {\scriptsize E} -- (85:0)  node [above] {\scriptsize F} arc (85:11:0.5) node [below, right] {\scriptsize D}  -- (0,-0.5); $=$ 
\begin{tikzpicture}[line width=2pt]
\fill[blue!70] (0,-0.5) node [left] {\scriptsize E} -- (85:0)  node [above] {\scriptsize F} arc (85:11:0.5)  -- (0,-0.5); 
\fill[fill=yellow] (0,-0.5) -- (-41:0.6) arc (11:-14:0.5) -- cycle;
\fill[black] (-41:0.6) node [right] {\scriptsize C};
\end{tikzpicture};

часть равна целому, что невозможно.

$\therefore$ \tikz\draw[line width=2pt] (0,0.2cm) node [above] {\scriptsize A} -- (1.5cm,0.2cm) node [above]  {\scriptsize C}; и \tikz\draw[line width=2pt, draw=red] (0,0.2cm) node [above] {\scriptsize B} -- (1.5cm,0.2cm) node [above]  {\scriptsize D}; не делят друг друга пополам.

\begin{flushright}
ч. т. д.
\end{flushright}
\end{center}
\end{minipage}
\end{document}